\documentclass[twocolumn,a4paper,11pt]{article}
\usepackage[]{times}
\usepackage[utf8]{inputenc}
\usepackage[IL2]{fontenc}
\usepackage[czech]{babel}
\usepackage[]{alltt}
\usepackage[hidelinks]{hyperref}
\usepackage[]{dirtytalk}
\usepackage[usenames,dvipsnames]{color}
\usepackage[left=1.4cm,text={18.2cm, 25.2cm},top=2.3cm]{geometry}
\usepackage{amsmath}
\usepackage{amsfonts}
\usepackage{amsthm}

%FOR_DIFF_PURPOSES
%\usepackage[]{xcolor}
%\color{red}

\newtheorem{defi}{Definice}
\newtheorem{veta}{Věta}

\begin{document}
\begin{titlepage}
    \begin{center}
        {\Huge \textsc{Vysoké učení technické v~Brně} \\[0.5em]} {\huge \textsc{Fakulta informačních technologií}} \\
        \vspace{\stretch{0.382}}
        {\LARGE Typografie a publikování\,--\,2. projekt \\[0.4em] Sazba dokumentů a matematických výrazů }\\
        \vspace{\stretch{0.612}}
    \end{center}

    {\Large 2023 \hfill Roland Schulz (xschul06)}
    \thispagestyle{empty}
\end{titlepage}

\pagenumbering{arabic}

\section*{Úvod}
\label{sec:intro}
V~této úloze si vyzkoušíme sazbu titulní strany, matematických vzorců, prostředí a dalších textových struktur obvyklých pro technicky zaměřené texty\,--\,například Definice~\ref{def:ZA} nebo rovnice~\eqref{eq:3} na straně~\pageref{sec:rovnice}. Pro vytvoření těchto odkazů používáme kombinace příkazů \verb|\label|, \verb|\ref|, \verb|\eqref| a \verb|\pageref|. Před odkazy patří nezlomitelná mezera. Pro zvýrazňování textu jsou zde několikrát použity příkazy \verb|\verb| a \verb|\emph|.

Na titulní straně je použito prostředí \verb|titlepage| a sázení nadpisu podle optického středu s~využitím \emph{přesného} zlatého řezu. Tento postup byl probírán na přednášce. Dále jsou na titulní straně použity čtyři různé velikosti písma a mezi dvojicemi řádků textu je použito odřádkování se zadanou relativní velikostí 0,5\,em a 0,4\,em\footnote{Nezapomeňte použít správný typ mezery mezi číslem a jednotkou.}.

\section{Matematický text}
\label{sec:mattext}
V~této sekci se podíváme na sázení matematických symbolů a výrazů v~plynulém textu pomocí prostředí \verb|math|.
Definice a věty sázíme pomocí příkazu \verb|\newtheorem| s~využitím balíku \verb|amsthm|.
Někdy je vhodné použít konstrukci \verb|${}$| nebo \verb|\mbox{}|, která říká, že (matematický) text nemá být zalomen.
\begin{defi}
\label{def:ZA}
\textup{Zásobníkový automat} (ZA) je definován jako sedmice tvaru $A = (Q, \Sigma, \Gamma, \delta, q_0 , Z_0 , F)$, kde:
\begin{itemize}
    \item $Q$ je konečná množina \textup{vnitřních (řídicích) stavů},
    \item $\Sigma$ je konečná \textup{vstupní abeceda},
    \item $\Gamma$ je konečná \textup{zásobníková abeceda},
    \item $\delta$ je \textup{přechodová funkce} $Q \times (\Sigma \cup \{\epsilon\}) \times \Gamma \rightarrow 2^{Q \times \Gamma^*}$,
    \item $q_0 \in Q$ je \textup{počáteční stav,} $Z_0 \in \Gamma$ je \textup{startovací symbol zásobníku} a $F \subseteq Q$ je množina \textup{koncových stavů}.
\end{itemize}
\end{defi}
\par Nechť $P = (Q,\Sigma,\Gamma,\delta,q_0,Z_0,F)$ je ZA. \emph{Konfigurací} nazveme trojici $(q,w,\alpha) \in Q \times \Sigma^*\times \Gamma^*$, kde $q$ je aktuální stav vnitřního řízení, $w$ je dosud nezpracovaná část vstupního řetězce a $\alpha = Z_{i_1}Z_{i_2}\dots Z_{i_k}$ je obsah zásobníku.

\subsection{Podsekce obsahující definici a větu}
\begin{defi}
    \label{def:retezec}
    \textup{Řetězec $w$ nad abecedou $\Sigma$ je přijat ZA} $A$ jest\-li\-že $(q_0,w,Z_0) \underset{A}{\overset{*}{\vdash}} (q_F,\epsilon,\gamma)$ pro nějaké $\gamma \in \Gamma^*$ a $q_F \in F$. Množina $L(A) = \{w \mid w \text{ je přijat ZA A}\} \subseteq \Sigma^*$ je \textup{jazyk přijímaný ZA} $A$.
\end{defi}

\begin{veta}Třída jazyků, které jsou přijímány ZA, odpovídá \textup{bezkontextovým jazykům}.\end{veta}

\section{Rovnice}
\label{sec:rovnice}
Složitější matematické formulace sázíme mimo plynulý text pomocí prostředí \verb|displaymath|. Lze umístit i několik výrazů na jeden řádek, ale pak je třeba tyto vhodně oddělit, například příkazem \verb|\quad|.
\[1^{2^3}\neq \Delta^1_{\Delta^2_{\Delta^3}}\quad y^{11}_{22}-\sqrt[9]{x+\sqrt[7]{y}}\quad x > y_1\leq y^2\]
V~rovnici~\eqref{eq:2} jsou využity tři typy závorek s~různou \emph{explicitně} definovanou velikostí. Také nepřehlédněte, že nasledující tři rovnice mají zarovnaná rovnítka, a použijte k~tomuto účelu vhodné prostředí.
\begin{eqnarray}
    -\cos^2\beta & = & \frac{\frac{\frac{1}{x}+\frac{1}{3}}{y}+1000}{\prod\limits_{j=2}^8 q_j} \label{eq:1} \\ 
    \biggl(\Bigl\{ b\star\bigl[3 \div 4\bigr]\circ a\Bigr\}^\frac{2}{3}\biggr) & = & \log_{10} x \label{eq:2} \\
    \int_a^b f(x)\,\mathrm{d}x & = & \int_c^d f(y)\,\mathrm{d}y \label{eq:3}
\end{eqnarray}
V~této větě vidíme, jak vypadá implicitní vysázení limity $\lim_{m \to \infty} f(m)$ v~normálním odstavci textu. Podobně je to i s~dalšími symboly jako $\bigcup_{N\in\mathcal{M}} N$ či $\sum^{m}_{i=1} x^2_i$.
S~vynucením méně úsporné sazby příkazem \verb|\limits| budou vzorce vysázeny v~podobě $\lim\limits_{m\to \infty} f(m)$ a $\sum\limits_{i=1}^{m} x^{4}_{i}$.

\section{Matice}
\label{sec:matice}
Pro sázení matic se velmi často používá prostředí \verb|array| a závorky (\verb|\left|, \verb|\right|).
\[
\mathbf{B} = \left|
\begin{array}{cccc}
    b_{11} & b_{12} & \cdots & b_{1n} \\
    b_{21} & b_{22} & \cdots & b_{2n} \\
    \vdots & \vdots & \ddots & \vdots \\
    b_{m1} & b_{m2} & \cdots & b_{mn} \\
\end{array}
\right|
= \left|\begin{array}{cc}
    t & u \\
    v & w \\
\end{array}
\right| = tw-uv
\]
\[
\mathbb{X} = \mathbf{Y} \Longleftrightarrow \left[
\begin{array}{ccc}
    & \Omega + \Delta & \hat{\psi} \\
    \vec{\pi} & \omega & \\
\end{array}
\right] \neq 42
\]
\par Prostředí \verb|array| lze úspěšně využít i jinde, například na pravé straně následující rovnice.
Kombinační číslo na levé straně vysázejte pomocí příkazu \verb|\binom|.
\[
\binom{n}{k} = \left\{ 
\begin{array}{cl}
    0 & \text{pro } k<0 \\
    \frac{n!}{k!(n-k)!} & \text{pro } 0\leq k~\leq n \\
    0 & \text{pro } k>0 \\
\end{array}
\right.
\]

\end{document}